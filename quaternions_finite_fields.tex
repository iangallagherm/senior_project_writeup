% LaTex file for senior project paper investigating the structure of commutative
% sub-algebras of the generalized quaternions over finite fields.
%   Author:     Ian Gallagher
%   Advisor:    Eric Brussel
%   School:     Cal Poly San Luis Obispo
\documentclass{amsart}

\usepackage{amsmath, amsthm, amssymb}
\usepackage{xcolor}
\usepackage{general}

\title{Investigating the Structure of $M_2(\bF_q)$}
\author{Ian Gallagher \\ Dr. Eric Brussel}
\date{\today}

\begin{document}

\maketitle

\begin{abstract}
    We present and prove a count of the maximal commutative sub-algebras of
    $M_2(\bF_q)$, as well as counts for the individual isomorphism classes. 
\end{abstract}

\section{Introduction}
For a finite field of $q = p^m$ elements, it is possible to count the number of
vector subspaces of ${\bF_q}^n$ of a given dimension. These counts arise in
problems involving the number of points of $\bP^n_q$, the Grassmannian,
and further generalizations. % Might want citations here

This paper is meant to address a similar problem. Namely, the structure and 
count of the maximal commutative sub-algebras of $M_2(\bF_q)$. Such
sub-algebra's...

\section{Identifying maximal commutative sub-algebras}


\section{Planes in $M_2(\bF_q)$}


\section{Plane Counts}

\begin{theorem}
    $M_2(\bF_q)$ has $q^2 + q + 1$ unique 2D commutative subalgebras.
\end{theorem}

Proof outline notes:
\begin{itemize}
    \item Each plane has $q^2$ elements.
    \item Each plane shares the $q$ elements of $\bF_q$.
    \item Each plane has trivial intersection.
    \item The planes cover all of $M_2(\bF_q)$.
\end{itemize}

\begin{proof}
    Let $E_x$ for $x \in M_2(\bF_q) - \bF_q$ be the commutative subalgbra created by the span of $1$ and $x$.
    Label $N = |\{E_x \mid x \in M_2(\bF_q) - \bF_q \}|$.

    \begin{align*}
        N(q^2 - q) + q &= q^4 \\
        N(q-1) + 1 &= q^3 \\
        N(q-1) &= q^3 - 1 \\
        N &= q^2 + q + 1
    \end{align*}
    A 2D commutative subalgebra $E$ must be such that $|E| = p^2$ and $\bF_q \subseteq E$.  
    
\end{proof}

\begin{lemma}
    Let $A \in M_2(k)$. Let $S = \det(kA)$. Then 
    \begin{align*}
        S =
        \begin{cases}
            \{0\}                     &\text{if } \det(A) = 0 \\
            k^{\times2}              &\text{if } \det(A) \in k^{\times2} \\
            k^{\times} - k^{\times2} &\text{if } \det(A) \in k^{\times} - k^{\times2}
        \end{cases}
    \end{align*}
%    Moreover, for elements $B, C \in kA$ if $\det(B) = \det(C) \neq 0$ then $B = C$ or $B = -C$.
\end{lemma}
\begin{proof}
    Let $A \in M_2(k)$. Then $\det(\lambda A) = \det(\lambda I)\det(A) = \lambda^2\det(A)$ where $\lambda \in k$. 
    \begin{Case}
        \item Suppose $\det(A) = 0$. Then $\lambda^2\det(A) = 0$ for all $\lambda \in k$ and we have $S = 0$.
        \item Suppose $\det(A) = \alpha^2$ where $\alpha \in k^{\times}$. Then $\lambda^2\det(A) = \lambda^2\alpha^2 = (\lambda\alpha)^2 \in k^{\times2}$. It follows that $S \subseteq k^{\times2}$. Now, let $\sigma \in k^{\times}$. Then $\det(\frac{\sigma}{\alpha}A) = \frac{\sigma^2}{\alpha^2}\alpha^2 = \sigma^2$. So $\sigma^2 \in S$ and we have shown $k^{\times2} \subseteq S$. Therefore, $S = k^{\times2}$.
        \item Suppose $\det(A) = \beta$ where $\beta \in k^{\times} - k^{\times2}$. Then $\lambda^2\det(A) = \lambda^2\beta \in k^{\times} - k^{\times2}$. It follows that $S \subseteq k^{\times} - k^{\times2}$. Since $|k^{\times}/k^{\times2}| = 2$, if $\sigma, \gamma \notin k^{\times2}$, we know there exists $\lambda \in k^{\times}$ such that $\sigma\lambda^2 = \gamma$. So $k^{\times} - k^{\times2} \subseteq S$ is clear and we have shown $S = k^{\times} - k^{\times2}$.
    \end{Case}
\end{proof}

\begin{lemma}
    Consider the following elements of $M_2(\bF_q)$.
    \begin{align*}
        x &= 
        \begin{pmatrix}
            0 & 1 \\
            1 & 0
        \end{pmatrix} \\
        y &=
        \begin{pmatrix}
            0 & 0 \\
            1 & 0
        \end{pmatrix}
    \end{align*}
    Under the conjugation action of $GL_2(\bF_q)$ on $M_2(\bF_q)$, 
    \begin{align*}
        |\mathcal{O}_x| &= q(q + 1) \\
        |\mathcal{O}_y| &= (q + 1)(q - 1)
    \end{align*}
\end{lemma}
\begin{proof}
    We may compute the size of the orbits of $x, y$ by first computing the size of their stabilizers, $G_x, G_y$, and then applying the orbit coset correspondence theorem.
    Now,
    \begin{align*}
        G_x &= \{A \in GL_2(\bF_q) \mid A \cdot x = x\} \\
            &= \{A \in GL_2(\bF_q) \mid AxA^{-1} = x\} \\
            &= \{A \in GL_2(\bF_q) \mid Ax = xA\} 
    \end{align*}
    Inevitably, matrices of the form $sx + tI$ commute with $x$ for $s,t \in \bF_q$, so if $sx + tI \in GL_2(\bF_q)$, then $sx + tI \in G_x$. These are also the only possible matrices in $G_x$ since these are the elements of the maximal commutative subalgebra containing $x$ {\color{blue} requires result citation/prior inclusion}. It now suffices to determine when $\det(sx + tI) = 0$.
    \begin{equation*}
        \det(sx+tI) =
        \begin{vmatrix}
            t & s \\
            s & t
        \end{vmatrix} = t^2 - s^2
    \end{equation*}
    Therefore, $\det(sx + tI) = 0$ when $t^2 = s^2$, so when $t = \pm s$. We have thus found precisely $q^2 - 2q + 1 = (q-1)^2$ elements of $G_x$. It follows that,
    \begin{equation*}
        [G:G_x] = \frac{(q^2-1)(q^2-q)}{(q-1)^2} = q(q-1)
    \end{equation*}
    By the orbit coset correspondence theorem {\color{blue} should cite/include prior} we have that $\#\operatorname{orbit}(x) = q(q + 1)$

    Similarly we have $G_y = \{A \in GL_2(\bF_q) | Ay = yA\}$ and the only possible elements of $G_y$ are those of the form $sy + tI$ for $s,t \in \bF_q$ where $sy+tI \in GL_2(\bF_q)$.
    \begin{equation*}
        \det(sy+tI) =
        \begin{vmatrix}
            t & 0 \\
            s & t
        \end{vmatrix} = t^2
    \end{equation*}
    So $\det(sy + tI) = 0$ if and only if $t = 0$. There are then $q^2 - q = q(q-1)$ elements of $G_y$. Therefore,
    \begin{equation*}
        [G:G_y] = \frac{(q^2-1)(q^2-q)}{q^2 - 1} = q(q-1)
    \end{equation*}
    and we have $\#orbit(y) = q^2 - 1$

    Now, referring to {\color{blue} cite previous lemma using Cref later} we know that each plane
\end{proof}

\begin{theorem}
    \begin{align}
        |[\bF_{q^2}]|                  &= \binom{q}{2} \\
        |[\bF_q \times \bF_q]|         &= \binom{q + 1}{2} \\[0.5em]
        |[{\bF_q^2}_{nil}]|            &= q + 1
    \end{align}
\end{theorem}
$x, y$ are the rational canonical forms for matrices with minimum polynomials $X^2 - 1$ and $X^2$ respectively.




\end{document}
