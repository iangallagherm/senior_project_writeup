% LaTex file for senior project paper investigating the structure of commutative
% sub-algebras of the generalized quaternions over finite fields.
%   Author:     Ian Gallagher
%   Advisor:    Eric Brussel
%   School:     Cal Poly San Luis Obispo
\documentclass{amsart}

\usepackage{amsmath, amsthm, amssymb}
\usepackage{xcolor}
\usepackage{general}

\title{Investigating the Structure of $M_2(\bF_q)$}
\author{Ian Gallagher \\ Dr. Eric Brussel}
\date{\today}

\begin{document}

\maketitle

\begin{abstract}
    We present and prove a count of the maximal commutative sub-algebras of
    $M_2(\bF_q)$, as well as counts for the individual isomorphism classes. 
\end{abstract}

\section{Introduction}
For a finite field of $q = p^m$ elements, it is possible to count the number of
vector subspaces of ${\bF_q}^n$ of a given dimension. These counts arise in
problems involving the number of points of $\bP^n_q$, the Grassmannian,
and further generalizations. % Might want citations here

This paper is meant to address a similar problem. Namely, the structure and 
count of the maximal commutative sub-algebras of $M_2(\bF_q)$. Such
sub-algebra's...

\section{Identifying maximal commutative sub-algebras}


\section{Planes in $M_2(\bF_q)$}


\section{Plane Counts}

\begin{lemma}
    Let $A \in M_2(k)$. Let $S = \det(kA)$. Then 
    \begin{align*}
        S =
        \begin{cases}
            \{0\}                     &\text{if } \det(A) = 0 \\
            k^{\times^2}              &\text{if } \det(A) \in k^{\times^2} \\
            k^{\times} - k^{\times^2} &\text{if } \det(A) \in k^{\times} - k^{\times^2}
        \end{cases}
    \end{align*}
%    Moreover, for elements $B, C \in kA$ if $\det(B) = \det(C) \neq 0$ then $B = C$ or $B = -C$.
\end{lemma}
\begin{proof}
    Let $A \in M_2(k)$. Then $\det(\lambda A) = \det(\lambda I)\det(A) = \lambda^2\det(A)$ where $\lambda \in k$. 
    \begin{Case}
        \item Suppose $\det(A) = 0$. Then $\lambda^2\det(A) = 0$ for all $\lambda \in k$ and we have $S = 0$.
        \item Suppose $\det(A) = \alpha^2$ where $\alpha \in k^{\times}$. Then $\lambda^2\det(A) = \lambda^2\alpha^2 = (\lambda\alpha)^2 \in k^{\times^2}$. It follows that $S \subseteq k^{\times^2}$. Now, let $\sigma \in k^{\times}$. Then $\det(\frac{\sigma}{\alpha}A) = \frac{\sigma^2}{\alpha^2}\alpha^2 = \sigma^2$. So $\sigma^2 \in S$ and we have shown $k^{\times^2} \subseteq S$. Therefore, $S = k^{\times^2}$.
        \item Suppose $\det(A) = \beta$ where $\beta \in k^{\times} - k^{\times^2}$. Then $\lambda^2\det(A) = \lambda^2\beta \in k^{\times} - k^{\times^2}$. It follows that $S \subseteq k^{\times} - k^{\times^2}$. Since $|k^{\times}/k^{\times^2}| = 2$, if $\sigma, \gamma \notin k^{\times^2}$, we know there exists $\lambda \in k^{\times}$ such that $\sigma\lambda^2 = \gamma$. So $k^{\times} - k^{\times^2} \subseteq S$ is clear and we have shown $S = k^{\times} - k^{\times^2}$.
    \end{Case}
\end{proof}

\begin{theorem}
    This is a theorem.
    \begin{align}
        |[\bF_{q^2}]|       &= \binom{q}{2} \\
        |[\bF_q^2]|         &= \binom{q + 1}{2} \\
        |[{\bF_q^2}_{nil}]| &= q + 1
    \end{align}
\end{theorem}
\begin{theorem}
    We shall consider the conjugation action of $GL_2(\bF_q)$ on $M_2(\bF_q)$ for the matrices,
    \begin{align*}
        x &= 
        \begin{pmatrix}
            0 & 1 \\
            1 & 0
        \end{pmatrix} \\
        y &=
        \begin{pmatrix}
            0 & 0 \\
            1 & 0
        \end{pmatrix}
    \end{align*}
    These are the rational canonical forms for matrices with minimum polynomials $X^2 - 1$ and $X^2$ respectively.
    We want to compute the size of their orbits, $G_x, G_y$, using the orbig coset correspondence theorem.
    Now,
    \begin{align*}
        G_x &= \{A \in GL_2(\bF_q) | A \cdot x = x\} \\
            &= \{A \in GL_2(\bF_q) | AxA^{-1} = x\} \\
            &= \{A \in GL_2(\bF_q) | Ax = xA\} 
    \end{align*}
    Clearly, matrices of the form $sx + tI$ for $s,t \in \bF_q$, so if $sx + tI \in GL_2(\bF_q)$, then $sx + tI \in G_x$. These are also the only possible matrices in $G_x$ since these are the elements of the maximal commutative subalgebra containing $x$ {\color{blue} requires result citation/prior inclusion}. It now suffices to determine when $\det(sx + tI) = 0$.
    \begin{equation*}
        \det(sx+tI) =
        \begin{vmatrix}
            t & s \\
            s & t
        \end{vmatrix} = t^2 - s^2
    \end{equation*}
    Therefore, $\det(sx + tI) = 0$ when $t^2 = s^2$, so when $t = \pm s$. We have thus found precisely $q^2 - 2q + 1$ elements of $G_x$.

\end{theorem}




\end{document}
