% LaTex file for senior project paper investigating the structure of commutative
% subalgebras of the generalized quaternions over finite fields.
%   Author:     Ian Gallagher
%   Advisor:    Eric Brussel
%   School:     Cal Poly San Luis Obispo
\documentclass{amsart}

\usepackage{amsmath, amsthm, amssymb, amsrefs}
\usepackage[margin=1.5in]{geometry}
\usepackage{general}
\usepackage{xcolor}
\usepackage{cleveref}


\title{Investigating the Structure of $M_2(\bF_q)$}
\author{Ian Gallagher \\ Dr. Eric Brussel}
\date{June 11, 2021}

\begin{document}

\maketitle

\begin{abstract}
    We present and prove a count of the maximal commutative subalgebras of
    $M_2(\bF_q)$, as well as counts for the individual isomorphism classes,
    primarily by leveraging results concerning the structure of the generalized
    quaternions algebra over $\bF_q$. We also prove that the orbits of 
    the maximal commutative subalgebras are exactly the isomorphism classes,
    and discuss the limiting ratios of the counts as q increases.
\end{abstract}

\section{Introduction}
For a finite field of $q = p^m$ elements, $p \neq 2$, it is possible to count the number of
vector subspaces of ${\bF_q}^n$ of a given dimension. These counts arise in
problems involving the number of points of $\bP^n_q$, the Grassmannian,
and further generalizations. % Might want citations here

This paper is meant to address a similar problem. Namely, the structure and 
count of the \textit{maximal commutative} subalgebras of $M_2(\bF_q)$, the vector space $\bF_q^4$ with matrix multiplication. Such commutative subalgebras are themselves $2$-dimensional vector subspaces with the inherited multiplication of $M_2(\bF_q)$. We find that there are three isomorphism classes of maximal commutative subalgebras and give explicit formulas for the size of the isomorphism classes based on the size of $\bF_q$.

%We also relate the limiting ratio of these counts to similar ratios that can be computed for $M_2(\bR)$.

For the duration of the paper, $q = p^n$ for some prime $p \neq 2$.

\section{Generalized Quaternions and $M_2(\bF_q)$}
Here, we introduce an alternative formulation of $M_2(\bF_q)$ based on the generalized quaternions, an abstraction of the well-known quaternion algebra. This formulation lends itself nicely to a more general result detailing the structure of commutative subalgebras of $M_2(\bF_q)$.

\begin{defn}[Generalized Quaternions]
    Let $\bF$ be a field, $\operatorname{char}(\bF) \neq 2$, and fix $a,b \in \bF-\{0\}$. Then define the \textit{generalized quaternions} as
\begin{gather*}
    A_{a,b}(\bF) = \{t + xi + yj + zk : t,x,y,z \in \bF\} \\
    \text{ where }i^2 =a, j^2 = b, \text{ and } ij = -ji=k
\end{gather*}
\end{defn}

Analogous to the complex numbers, for $q = t + xi + yj + zk$ we denote the quaternion conjugate $\overline{q} = t - xi - yj - zk$ and we will call the additive subspace 
\begin{align*}
    \{ xi + yj + zk : x, y, z \in \bF \} \subseteq A_{a,b}(\bF)
\end{align*}
the pure imaginary elements,

\begin{theorem}
    $A_{a,b}(\bF)$ is a central simple algebra over $\bF$. That is, it is a finite dimensional \bF-algebra with center \bF and no nonzero proper two-sided ideals.
\end{theorem}
\begin{proof}
    Denote $A = A_{a,b}(\bF)$. If $q = t + xi + yj + zk$, then it is easy to show $iqi = a(t + xi -yj -zk)$, $jqj = b(t - xi + yj -zk)$, $kqk = -ab(t-xi-yj + zk)$. We then have:
    \begin{enumerate}
        \item The center of $A$ is \bF: Using $i^2 = a$, $j^2 = b$, and $k^2 = -ab$, it follows that $q$ commutes with $i, j, k$ if and only if $q = t \in \bF$.
        \item A has no nonzero proper two-sided ideals: It suffices to show that $AqA = A$ for any nonzero $q = t + xi + yj + zk$. Since $AqA$ contains $iqi$, $jqj$, and $kqk$, it contains every linear combination of them, and since $2$, $a$, and $b$ are invertible in \bF, we can use this to show $t \in AqA$. If t is nonzero, then it is a $2$-sided unit, and since any ideal containing a $2$-sided unit is all of $A$, we are done. If $t = 0$ then it follows quickly that $x, y, z$ are all in $AqA$ and since $q$ is nonzero, one of them is a $2$-sided unit, so $AqA = A$. Therefore, in every case, $AqA = A$, so $A$ has no nonzero proper $2$-sided ideals.
    \end{enumerate}
\end{proof}

\begin{defn}[Generalized Quaternion Norm]
    We define the \textit{generalized quaternion norm} from $A_{a,b}(\bF)$ to \bF to be the function 
    \begin{align*}
        N_{a,b}:  A_{a,b}(\bF) & \rightarrow \bF  \\
        q & \mapsto q \overline{q} = t^2 -ax^2 -by^2 +abz^2 
    \end{align*}
\end{defn}
For $q, w \in A_{a,b}(\bF)$, it is not difficult to show that $\overline{qw} = \overline{w} \overline{q}$. From this we note that the norm is multiplicative
\begin{align*}
    N_{a,b}(qw) = qw\overline{qw} = qw\overline{w}\overline{q}= q\overline{q}w\overline{w} = N_{a,b}(q)N_{a,b}(w)
\end{align*}
where the third equality holds since $w\overline{w} \in \bF$ and therefore commutes with $\overline{q}$.

It turns out that $M_2(\bF_q)$ is the only isomorphism class of generalized quaternions over $\bF_q$ but this fact is not immediately obvious. We will use the following two results to prove this.

\begin{theorem}[Wedderburn~\cite{DandF}*{pg. 854}]\label{wedderburn}
    Let $\bF$ be a field and $A$ a central simple algebra. Then $A \simeq M_n(D)$ for some $\bF$-division algebra $D$.
\end{theorem}

Refer to ~\cite{Henderson65} for a short proof of Wedderburn's Theorem.

\begin{theorem}[Chevalley-Warning~\cite{Serre}*{pg. 5}]\label{c-warning}
    Let $f_{\alpha} \in \bF_q[X_1, \dots, X_n]$ be polynomials in $n$ variables such that $\sum_{\alpha} \deg f_{\alpha} < n$ and let $V$ be the set of their common zeros in $\bF_q^n$. One has 
    \begin{align*}
        | V | \equiv 0 \pmod{p}
    \end{align*}
\end{theorem}

With these theorems, we now give a proof of the structure of the generalized quaternions over $\bF_q$.

\begin{theorem}\label{gen-quat-finite}
    $A_{a,b}(\bF_q) \cong M_2(\bF_q)$ 
\end{theorem}
\begin{proof}
    We first make use of \Cref{c-warning}. Let
    \begin{align*}
        f &= N_{a,b}(X_1 + iX_2 + jX_3 + kX_4) \\
            &= X_1^2 -aX_2^2 -bX_3^2 + abX_4^2 \\
            &\in \bF_q[X_1, X_2, X_3, X_4]
    \end{align*}
    Here, $\deg f = 2 < 4$, so Chevalley-Warning holds and $|V| \equiv 0 \pmod{p}$, where $V$ is the set of zeros of $f$. Note that $f(0,0,0,0) = 0$, so we must have $p$ divides $|V|$. Therefore, $f$ has a nontrivial zero.

    It follows that there exists $q \in A_{a,b}(\bF_q)$, $q \neq 0$, such that $N_{a,b}(q) = q \overline{q} = 0$. Therefore, q is a nontrivial zero divisor and $A_{a,b}(\bF_q)$ is not a $\bF_q$-division algebra. By \Cref{wedderburn} we must have $A_{a,b}(\bF_q) \cong M_n(D)$ for some $\bF_q$-division algebra $D$ and natural number $n$. Since $A_{a,b}(\bF_q)$ is not a division algebra, we know $n\neq1$. $A_{a,b}(\bF_q)$ has $\bF_q$-dimension $4$, so $n = 2$ and $D$ must have $\bF_q$-dimension 1, so is isomorphic to $\bF_q$.Therefore, $A_{a,b}(\bF_q) \cong M_2(\bF_q)$. 
\end{proof}

\section{Commutative Planes in $M_2(\bF_q)$}
    \begin{defn}
        For a matrix $A \in M_2(\bF_q)$, define
        \begin{enumerate}
            \item $A$'s {\it characteristic polynomial} $p_A(X) = X^2 -\operatorname{tr}(A)X + \det(A)$.
            \item $A$'s {\it minimal polynomial} $m_A(X)$ is the unique monic polynomial of smallest degree such that $m_A(A) = 0$.
        \end{enumerate}
    \end{defn}

    By the Cayley-Hamilton Theorem~\cite{DandF}*{pg. 478}, we know that $m_A(X)$ divides $p_A(X)$. If $A \in \bF_q\cdot I\subset M_2(\bF_q)$, then $m_A(X) = X - A$. If $A \in M_2(\bF_q)-\bF_q\cdot I$ then $m_A(X)$ does not have degree 1, and thus $m_A(X) = p_A(X)$  by Cayley-Hamilton. 
    
    \begin{defn}
        A {\it commutative plane of $M_2(\bF_q)$} is a $2$-dimensional commutative subalgebra of $M_2(\bF_q)$.
    \end{defn}

    \begin{theorem}
        Every commutative plane of $M_2(\bF_q)$ has the form $\bF_q[A]$ for some $A \in M_2(\bF_q)$, and belongs to one of three isomorphism classes, depending on $m_A(X)$ as follows.
        \begin{enumerate}
            \item If $m_A(X)$ is irreducible, $\bF_q[A] \cong \bF_{q^2}$.
            \item If $m_A(X)$ is reducible but separable, $\bF_q[A] \cong \bF_q \times \bF_q$.
            \item If $m_A(X)$ is reducible but inseparable, $\bF_q[A] \cong {\bF_q^2}_{nil}$.
        \end{enumerate}
    \end{theorem}

    \begin{proof}
        Let $K$ be a commutative plane of $M_2(\bF_q)$. Then there exists $A \in K - \bF_q$. Define the {\it evaluation map} $\varepsilon_A : \bF_q[X] \rightarrow M_2(\bF_q)$ taking $f(X)$ to $f(A)$, which is a $\bF_q$-algebra homomorphism with image $\bF_q[A] \subseteq M_2(\bF_q)$. The kernel is nontrivial, since $M_2(\bF_q)$ has finite $\bF_q$-dimension and $\bF_q[X]$ does not. As $m_A(A) = 0$ by definition, $m_A(X) \subseteq \ker(\varepsilon_A)$. Since $\bF_q[X]$ is a principal ideal domain, and the minimal polynomial is defined to be the smallest degree polynomial with $A$ as a root, $\ker(\varepsilon_A) = (m_A(X))$. By the First Isomorphism Theorem,
        \begin{align*}
            \frac{\bF_q[X]}{(m_A(X))} \cong \bF_q[A]
        \end{align*}
        Since $A \in M_2(\bF_q) - \bF_q$, we know $m_A(X) = p_A(X)$ is degree $2$ and $\bF_q[A]$ has $\bF_q$-dimension $2$, hence $K=\bF_q[A]$. We conclude that all possible commutative planes are of the form
        \begin{align*}
            \frac{\bF_q[X]}{(m_A(X))}
        \end{align*}
        for some $A \in M_2(\bF_q) - \bF_q$. Now the following is an easy exercise.
        \begin{enumerate}
            \item If $m_A(X)$ is irreducible, $\bF_q[A] \cong \bF_{q^2}$.
            \item If $m_A(X)$ is reducible but separable, $\bF_q[A] \cong \bF_q \times \bF_q$.
            \item If $m_A(X)$ is reducible but inseparable, $\bF_q[A] \cong {\bF_q^2}_{nil}$.
        \end{enumerate}
        Thus the set of commutative planes of $M_2(\bF_q)$ forms exactly three isomorphism classes based on $m_A(X)$, as stated.
    \end{proof}

    The commutative subalgebras generated when $m_A(X)$ is reducible can each be considered degenerate cases. 
    
    When $m_A(X)$ is separable, we have the presence of nontrivial zero divisors. Since $m_A(X)$ is separable and reducible, we have $m_A(X) = (X - x)(X - y)$ for some $x,y \in \bF_q$, $x \neq y$. By the definition of the minimum polynomial, we must have that $(A - x)(A - y) = 0$. $A$ is not an element of $\bF_q$, so $(A - x), (A - y)$ are nonzero and are therefore nontrivial zero divisors.
    
    When $m_A(X)$ is inseparable, we have the presence of nontrivial nilpotent elements which square to $0$. Since $m_A(X)$ is inseparable and reducible, we have $m_A(X) = (X - x)^2$ for some $x \in \bF_q$. By the definition of the minimal polynomial, we must have that $(A - x) ^ 2 = 0$. $A$ is not an element of $\bF_q$, so $A - x$ is nonzero and is therefore a nontrivial nilpotent element.
    
    It is worth noting that the case that $m_A(X)$ is both inseparable and irreducible does not require consideration since finite fields are perfect.

    
    

\section{Identifying maximal commutative subalgebras}

We have shown that each matrix $A \in M_2(\bF_q) - \bF_q$ generates a commutative plane $\bF_q[A]$, and that this plane belongs to one of three isomorphism classes.
Since there exist subalgebras of dimension 3, such as the set of upper triangular matrices, it is reasonable to ask whether this is the maximal commutative subalgebra of $M_2(\bF_q)$ containing $A$.

We will show the answer is no: there are no commutative subalgebras of dimension 3, and each commutative plane $\bF_q[A]$ is therefore maximal. To prove it, we return to the generalized quaternions formulation. We first introduce a useful vector notation for the generalized quaternions.

\begin{defn}[Generalized Quaternions, Vector Notation]
    Let $p, q \in A_{a,b}(\bF)$. Write $p = (s, \mathbf{v})$, and $q = (t, \mathbf{w})$, using the vector notation $\mathbf{v} = \langle v_1i, v_2j, v_3k \rangle$ and $\mathbf{w} = \langle w_1i, w_2j, w_3k \rangle$ with coefficients in $A_{a,b}(\bF)$. Modified dot and cross product formulas compute
    \begin{align*}
        \mathbf{v} \cdot_{a,b}  \mathbf{w} &= v_1w_1a + v_2w_2b -abv_3w_3 \\
        \mathbf{v} \times_{a,b} \mathbf{w} &=
            \left\langle\begin{vmatrix}
                v_2j & v_3k \\
                w_2j & w_3k
            \end{vmatrix},
            \begin{vmatrix}
                v_1i & v_3k \\
                w_1i & w_3k
            \end{vmatrix},
            \begin{vmatrix}
                v_1i & v_2j \\
                w_1i & w_2j
            \end{vmatrix}\right\rangle \\
                                            &= \langle (-v_2w_3 + v_3w_2)bi , (v_1w_3 - v_3w_1)aj, (v_1w_2 - v_2w_1)k \rangle
    \end{align*}
    Compare this to $\langle v_1, v_2, v_3 \rangle \times \langle w_1, w_2, w_3 \rangle = \langle v_2w_3 - v_3w_2, -v_1w_3 + v_3w_1, v_1w_2 - v_2w_1 \rangle$. Thus 
    \begin{align*}
        pq &= (st + v_1w_1a + v_2w_2b -abv_3w_3) + (sw_1 + tv_1 - v_2w_3b + v_3w_2b)i \\
           &\qquad + (sw_2 + tv_2 + v_1w_3a-v_3w_1a)j + (sw_3 + tv_3 + v_1w_2 - v_2w_1)k \\
           &= (st + \mathbf{v} \cdot_{a,b} \mathbf{w}, s\mathbf{w} + t\mathbf{v} + \mathbf{v} \times_{a,b} \mathbf{w})
    \end{align*}
\end{defn}

\begin{lemma}\label{anticommutativity}
    Let $p = (0, \mathbf{v})$, and $q = (0, \mathbf{w})$. Then $pq = qp$ if and only if $\mathbf{v} \times_{a,b} \mathbf{w} = 0$ if and only if $\langle v_1, v_2, v_3 \rangle \times \langle w_1, w_2, w_3 \rangle = 0$.
\end{lemma}
\begin{proof}
    We have $pq - qp = 2 \mathbf{v} \times_{a,b} \mathbf{w}$, and since $i, j, k$ are linearly independent and $\operatorname{char} k \neq 2$, $pq = qp$ if and only if $\mathbf{v} \times_{a,b} \mathbf{w} = 0$.

    Additionally, setting $\langle t_1, t_2, t_3 \rangle = \langle v_1, v_2, v_3 \rangle \times \langle w_1, w_2, w_3 \rangle$, we observe $\mathbf{v} \times_{a,b} \mathbf{w} = \langle (-bi)t_1, (-aj) t_2, (k) t_3 \rangle$. Also, $i, j, k$ are all not roots of the norm polynomial and are therefore units. 
    \begin{align*}
        N_{a,b}(i) &= i * (-i) = -a \\
        N_{a,b}(j) &= j * (-j) = -b \\
        N_{a,b}(k) &= k * (-k) = ab
    \end{align*}
    Since $i, j, k$ are units in $M_2(\bF_q)$, they cannot be zero divisors, and we have that $\mathbf{v} \times_{a,b} \mathbf{w} = 0$ if and only if $\langle v_1, v_2, v_3 \rangle \times \langle w_1, w_2, w_3 \rangle = 0$, as desired. 
\end{proof}

\begin{theorem}\label{commutative-planes}
    If $K \subseteq A_{a,b}(\bF)$ is a nontrivial commutative subalgebra, then $K = \bF[\mathbf{w}]$ for some nonzero pure imaginary $\mathbf{w} \in A_{a,b}(\bF)$, and $K$ has $\bF$-dimension $2$.
\end{theorem}
\begin{proof}
    If $q = (t, \mathbf{w}) \in K - \bF$, then $q-t = \mathbf{w}$ is in $\bF[q]$, hence $\bF[\mathbf{w}] = \bF[q] \subseteq K$. If $p = (s, \mathbf{v}) \in K$ then since $pq = qp$, $\mathbf{v}$ is a scalar multiple of $\mathbf{w}$ by \Cref{anticommutativity}. Therefore, $K \subseteq \bF[\mathbf{w}]$. It follows that $K = \bF[\mathbf{w}]$. Finally, since $w \notin \bF$, we must have $m_{\mathbf{w}}(X) = p_{\mathbf{w}}(X)$, so $\deg(m_{\mathbf{w}}(X)) = 2$ and this completes the proof.
\end{proof}

Leveraging the isomorphism between $A_{a,b}(\bF_q)$ and $M_2(\bF_q)$ proven in \Cref{gen-quat-finite}, we may now fully characterize the structure of maximal commutative subalgebras in $M_2(\bF_q)$.

\begin{cor}
    The three isomorphism classes of commutative planes are the maximal commutative subalgebras of $M_2(\bF_q)$.
\end{cor}
\begin{proof}
    $A_{a,b}(\bF_q) \cong M_2(\bF_q)$, so \Cref{commutative-planes} suffices to show that no commutative subalgebra of dimension 3 exists. Therefore, the commutative planes are the maximal commutative subalgebras of $M_2(\bF_q)$ as desired.
\end{proof}


%To address this we state, but do not prove, a related result by Jacobson.

%\begin{thm}[Jacobson~\cite{Jacobson44}]
%    For an arbitrary field $k$, the maximum number $N(n)$ of linearly independent commutative matrices of $n$ rows and columns is given by the formula $N(n) = \lfloor n^2/4 \rfloor + 1$.
%\end{thm}

%In the case of $M_2(\bF_q)$, $n=2$ and we have that $N(2) = \lfloor 2^2/4 \rfloor + 1 = 2$. Therefore, these planes containing A are indeed maximal.

\section{Plane Counts}

At this point we are ready to prove more rigorous statements concerning the structure of the commutative planes in $M_2(\bF_q)$, including computing the size of each isomorphism class of commutative planes, and the orbits of the commutative planes under the conjugation action of $GL_2(\bF_q)$. This then gives us a discrete ratio of the three plane types, as well as a limiting ratio as $q \rightarrow \infty$. We begin with the total count of commutative planes in $M_2(\bF_q)$. 

\begin{theorem}\label{total-count}
    $M_2(\bF_q)$ has $q^2 + q + 1$ unique commutative planes.
\end{theorem}
\begin{proof}
    By \Cref{commutative-planes}, each commutative plane is of the form $\bF_q[\mathbf{w}]$ for some pure imaginary $\mathbf{w} \in M_2(\bF_q)$, and is therefore uniquely determined by the line $\bF_q \mathbf{w}$ by \Cref{anticommutativity}. The number of commutative planes is then the same as the number of pure imaginary lines through the origin. Since there are $q^3 -1$ choices of pure imaginaries in $M_2(\bF_q)$ and $\mathbf{w}$ has $q-1$ nonzero scalar multiples in $\bF_q$, there must then be
    \begin{align*}
        \frac{q^3 - 1}{q - 1} = q^2 + q + 1
    \end{align*}
    unique commutative planes in $M_2(\bF_q)$. 
%
    %Let $E_x$ for $x \in M_2(\bF_q) - \bF_q$ be the commutative subalgebra created by the span of $1$ and $x$.
    %Label $\Sigma = \{E_x \mid x \in M_2(\bF_q) - \bF_q \}$ and set $N = |\Sigma|$, the number of unique 2D commutative subalgebras. Let $x, y \in M_2(\bF_q) - \bF_q$. $E_x, E_y$ are both two dimensional $\bF_q$ algebras so we know that $E_x \cap E_y$ is also a $\bF_q$-algebra. By dimension arguments, either $E_x = E_y$ or $E_x \cap E_y = \bF_q$. Since $\bigcup_{E \in \Sigma}E = M_2(\bF_q)$, $\{E_x - \bF_q \mid x \in M_2(\bF_q) - \bF_q \}$ is a partition of $M_2(\bF_q) - \bF_q$. The following then holds,
%
    %\begin{align*}
    %    N ( | E_x | - | \bF_q | ) &= | M_2(\bF_q) | - | \bF_q | \\
    %    N(q^2 - q) &= q^4 - q \\
    %    N(q-1) &= q^3 - 1 \\
    %    N &= q^2 + q + 1
    %\end{align*}
%    
    \end{proof} 
    We must then have that the cardinality of the isomorphism classes of commutative planes sum to this total plane count of $q^2 + q + 1$. Before presenting formulas for the counts of the individual isomorphism classes, we first demonstrate an equivalence between the isomorphism classes of commutative planes and their orbits under conjugation.

%    \begin{lemma}\label{square-quotient-cosets}
%    Let k be a field such that $\left[k : k^{\times2}\right] = 2$ and $A \in M_2(k)$. Define $S = \{\det( \lambda A ) \mid \lambda \in k^{\times} \}$.  \begin{align*}
%        S =
%        \begin{cases}
%            \{0\}                     &\text{if } \det(A) = 0 \\
%            k^{\times2}              &\text{if } \det(A) \in k^{\times2} \\
%            k^{\times} - k^{\times2} &\text{if } \det(A) \in k^{\times} - k^{\times2}
%        \end{cases}
%    \end{align*}
%    Moreover, for elements $B, C \in kA$ if $\det(B) = \det(C) \neq 0$ then $B = C$ or $B = -C$.
%\end{lemma}
%\begin{proof}
%    Let $A \in M_2(k)$. Then $\det(\lambda A) = \det(\lambda I)\det(A) = \lambda^2\det(A)$ where $\lambda \in k$.  If $\det(A) = 0$, then $\det(\lambda A) = 0$ for $\lambda \in k$ and $S = \{0\}$ follows. Otherwise, we know that the quotient group $k^{\times}/k^{\times2}$ has only the two cosets $k^{\times2}, k^{\times} - k^{\times2}$. So if $\det(A) \in k^{\times2}$, we have $S = \{\lambda^2 \det(A) \mid \lambda \in k^{\times}\} = k^{\times2}$ and similarly, if $\det(A) \in k^{\times} - k^{\times2}$, then $S = \{\lambda^2 \det(A) \mid \lambda \in k^{\times}\} = k^{\times} - k^{\times2}$.
%\end{proof}

\begin{theorem}\label{plane-orbits}
    The orbits of the commutative planes in $M_2(\bF_q)$ under the conjugation action of $GL_2(\bF_q)$ are precisely the three isomorphism classes of planes.
\end{theorem}
\begin{proof}
    Let $K, K'$ be commutative planes in $M_2(\bF_q)$ such that $K \cong K'$. That is, $K, K'$ are the same plane type. Let $A \in K$, $B \in K'$ be nonzero elements orthogonal to the identity. Then $K = \bF_q[A]$, $K' = \bF_q[B]$ and 
    \begin{align*}
        m_A(X) &= p_A(X) = X^2 + \alpha \\
        m_B(X) &= p_B(X) = X^2 + \beta
    \end{align*}
    where $\alpha = \det(A)$, $\beta = \det(B)$. Furthermore, we must have that $\alpha \bF_q^{\times2} = \beta \bF_q^{\times2}$ since $K, K'$ belong to the same plane type. It follows that there exists $\delta \in \bF_q^{\times}$ such that $\alpha = \delta^2 \beta$. Let $C = \delta B$. Then $C$ is a nonzero element of $K'$ orthogonal to the identity with $\bF_q[C] = \bF_q[B] = K'$ and 
    \begin{align*}
        \det(C) = \det(\delta I)\det(B) = \delta^2 \beta = \alpha
    \end{align*}
    and we have that $m_A(X) = m_C(X) = X^2 + \alpha$. Therefore
    \begin{align*}
        \left(\begin{matrix}
            0 & -\alpha \\
            1 & 0
        \end{matrix}\right)
    \end{align*}
    is the rational canonical form for both $A$ and $C$. Two matrices in $M_2(\bF)$ conjugate if and only if they have the same rational canoncial form~\cite{DandF}*{pg. 476}. Therefore $A$ and $C$ are conjugate, so the commutative planes $K = \bF_q[A]$ and $K' = \bF_q[C]$ are conjugate.
    
    We have thus shown that two commutative planes are conjugate if they are of the same isomorphism class. We also know that two conjugate commutative planes are isomorphic, so we may conclude that the orbits of the commutative planes of $M_2(\bF_q)$ under the conjugation action of $GL_2(\bF_q)$ are precisely the three isomorphism classes, as desired.
\end{proof}

To compute the individual count of the orbits of each commutative plane, we first must compute the size of the orbits of some distinguished elements of $M_2(\bF_q)$.

\begin{lemma}\label{orbit-sizes}
    Consider the following elements of $M_2(\bF_q)$.
    \begin{align*}
        P &= 
        \begin{pmatrix}
            0 & 1 \\
            1 & 0
        \end{pmatrix} \\
        Q &=
        \begin{pmatrix}
            0 & 0 \\
            1 & 0
        \end{pmatrix}
    \end{align*}
    Let $\mathcal{O}_P, \mathcal{O}_Q$ denote the orbits of $P, Q$ under the conjugation action of $GL_2(\bF_q)$ on $M_2(\bF_q)$. Then
    \begin{align*}
        |\mathcal{O}_P| &= q(q + 1) \\
        |\mathcal{O}_Q| &= (q + 1)(q - 1)
    \end{align*}
\end{lemma}
\begin{proof}
    We may compute the size of the orbits of $P, Q$ by first computing the size of their stabilizers, $G_P, G_Q$, and then applying the Orbit-Coset Correspondence Theorem \cite{DandF}*{pg. 114}.
    Now
    \begin{align*}
        G_P &= \{A \in GL_2(\bF_q) \mid A \cdot P = P\} \\
            &= \{A \in GL_2(\bF_q) \mid APA^{-1} = P\} \\
            &= \{A \in GL_2(\bF_q) \mid AP = PA\} 
    \end{align*}
    Since matrix multiplication is distributive, matrices of the form $sP + tI$ commute with $P$ for $s,t \in \bF_q$, so if $sP + tI \in GL_2(\bF_q)$, then $sP + tI \in G_P$. These are also the only possible matrices in $G_P$ since they are the elements of the maximal commutative subalgebra containing $P$ (\Cref{commutative-planes}). It now suffices to determine when $\det(sP + tI) = 0$.
    \begin{equation*}
        \det(sP+tI) =
        \begin{vmatrix}
            t & s \\
            s & t
        \end{vmatrix} = t^2 - s^2
    \end{equation*}
    Therefore, $\det(sP + tI) = 0$ when $t^2 = s^2$, so when $t = \pm s$. This eliminates the $q$ scalar multiples of $P + I$ and the $q$ scalar multiples of $P - I$, of which only $0$ is shared between them. We have thus found precisely $q^2 - 2q + 1 = (q-1)^2$ elements of $G_P$. It follows that
    \begin{equation*}
        [GL_2(\bF_q) : G_P] = \frac{(q^2-1)(q^2-q)}{(q-1)^2} = q(q + 1)
    \end{equation*}
    where $[GL_2(\bF_q) : G_P]$ is the index of $G_P$ in $GL_2(\bF_q)$. By the Orbit-Coset Correspondence Theorem we have that $|\mathcal{O}_P| = q(q + 1)$.

    Similarly we have $G_Q = \{A \in GL_2(\bF_Q) | AQ = QA\}$ and the only possible elements of $G_Q$ are those of the form $sQ + tI$ for $s,t \in \bF_q$ where $sQ+tI \in GL_2(\bF_q)$.
    \begin{equation*}
        \det(sQ+tI) =
        \begin{vmatrix}
            t & 0 \\
            s & t
        \end{vmatrix} = t^2
    \end{equation*}
    So $\det(sQ + tI) = 0$ if and only if $t = 0$. There are then $q^2 - q$ elements of $G_Q$. Therefore
    \begin{equation*}
        [GL_2(\bF_q) : G_Q] = \frac{(q^2-1)(q^2-q)}{q^2 - q} = q^2 - 1
    \end{equation*}
    and we have $|\mathcal{O}_Q| = q^2 - 1$
\end{proof}

Finally, we are ready to prove the following formulas for the counts of each type of commutative plane in $M_2(\bF_q)$. Note that the sum of the three commutative plane counts is $q^2 + q + 1$ as desired.

\begin{theorem}\label{plane-counts}
    The size of the orbits of the commutative planes in $M_2(\bF_q)$ under the conjugation action of $GL_2(\bF_q)$ are as follows
    \begin{align}
        |[\bF_{q^2}]|                   &= \binom{q}{2} \\
        |[\bF_q \times \bF_q]|          &= \binom{q + 1}{2} \\[0.5em]
        |[{\bF_q^2}_{nil}]|             &= q + 1
    \end{align}
\end{theorem}
\begin{proof}
    The elements $P, Q$ from \Cref{orbit-sizes} are the rational canonical forms for all matrices with minimal polynomials $X^2 - 1$ and $X^2$ respectively. Note again that two matrices in $M_2(\bF)$ are conjugate if and only if they have the same rational canonical form.

    Since $X^2 - 1 = (X + 1)(X - 1)$ is reducible and separable over $\bF_q$, and $\operatorname{char}(\bF_q) \neq 2$, all of the elements of $\mathcal{O}_P$ generate a commutative plane isomorphic to $\bF_q \times \bF_q$.

    Any copy of $K \subseteq M_2(\bF_q)$ isomorphic to $\bF_q \times \bF_q$ must also contain an element of $\mathcal{O}_P$. To see this, pick a nontrivial element $A \in K$ such that $A$ is orthogonal to the identity. Then $\operatorname{tr}(A) = 0$, and $m_A(X) = p_A(X) = X^2 + \det(A)$. We picked $A \in K$, which means that $m_A(X)$ must also be reducible and separable. It follows that $\det(A) = -\lambda^2$ for some $\lambda \in \bF_q$. Let $B = A/\lambda \in K$ and observe $\operatorname{tr}(B) = \operatorname{tr}(A)/\lambda = 0$, $\det(B) = \det(A)/\lambda^2 = -\lambda^2/\lambda^2 = -1$. Therefore, $m_B(X) = p_B(X) = X^2 -\operatorname{tr}(B) X + \det(B) = X^2 -1$ as desired.

    Furthermore, suppose $B' \in K$ has minimum polynomial $X^2 - 1$. Then $B'$ is orthogonal to the identity as well and $B' = \alpha B$ for some $\alpha \in \bF_q^{\times}$ due to $K$ being dimension $2$. We have $-1 = \det(B') = \alpha^2 \det(B) = -\alpha^2$, so $\alpha = \pm 1$ and $B' = \pm B$. We have therefore shown that each possible choice of $K$ contains exactly $2$ elements of $\mathcal{O}_P$. Applying \Cref{orbit-sizes},
    \begin{align*}
        |[\bF_q \times \bF_q]| = \frac{1}{2}|\mathcal{O}_P|
                  = \frac{q(q+1)}{2}
                  = \binom{q + 1}{2}
    \end{align*}
    Similarly, since $X^2$ is inseparable over $\bF_q$, all of the elements of $\mathcal{O}_Q$ generate a commutative subalgebra isomorphic to ${\bF_q^2}_{nil}$. Any isomorphic copy $K$ will contain exactly $q - 1$ elements of $\mathcal{O}_Q$. These are precisely the nontrivial elements of $K$ that are orthogonal to the identity. Applying \Cref{orbit-sizes},
    \begin{align*}
        |[{\bF_q^2}_{nil}]| = \frac{1}{q-1}|\mathcal{O}_Q|
                            = \frac{q^2-1}{q-1}
                            = q+1
    \end{align*}

    To show $|[\bF_{q^2}]| = \binom{q}{2}$, we apply \Cref{total-count} and compute.
    \begin{align*}
        |[\bF_{q^2}]| &= q^2 + q + 1 - |[\bF_q \times \bF_q]| - |[{\bF_q^2}_{nil}]| \\
                      &= q^2 + q + 1 - \binom{q+1}{2} - (q+1) \\
                      &= q^2 - \frac{q(q+1)}{2} \\
                      &= \frac{q(q-1)}{2} \\
                      &= \binom{q}{2}
    \end{align*}
    This completes the proof.
\end{proof}
From \Cref{plane-counts} we can also find the limiting ratios of these commutative planes as $q$ gets larger. This ratio is of interest partially due to its relationship to the ratio of the three commutative planes in $M_2(\bR)$, although we will not discuss this here.
\begin{cor}
    As $q \rightarrow \infty$, the ratio of commutative subalgebras of $M_2(\bF_q)$
    \begin{align*}
        |[\bF_{q^2}]| : |[\bF_q \times \bF_q]| : |[{\bF_q^2}_{nil}]|
    \end{align*}
    approaches $1 : 1 : 0$.
\end{cor}

\begin{proof}
    It suffices to show that the limit of $|[\bF_{q^2}]| / |[\bF_q \times \bF_q]|$ approaches $1$ and that the limit of  $|[{\bF_q^2}_{nil}]| / |[\bF_q \times \bF_q]|$ approaches $0$.
    \begin{align*}
        \lim\limits_{q \rightarrow \infty} \frac{|[\bF_{q^2}]|}{|[\bF_q \times \bF_q]|}
            = \lim\limits_{q \rightarrow \infty} \frac{\binom{q}{2}}{\binom{q+1}{2}}
            = \lim\limits_{q \rightarrow \infty} \frac{q(q-1)}{2}\frac{2}{q(q+1)}
            = \lim\limits_{q \rightarrow \infty} \frac{(q-1)}{(q+1)}
            = 1
    \end{align*}
    \begin{align*}
        \lim\limits_{q \rightarrow \infty} \frac{|[{\bF_q^2}_{nil}]|}{|[\bF_q \times \bF_q]|}
            = \lim\limits_{q \rightarrow \infty} \frac{q+1}{\binom{q+1}{2}}
            = \lim\limits_{q \rightarrow \infty} (q+1)\cdot\frac{2}{q(q+1)}
            = \lim\limits_{q \rightarrow \infty} \frac{2}{q}
            = 0
    \end{align*}
\end{proof}

\section{Further Work}
There are a variety of different directions of inquiry from which the present results can be taken. Of particular interest to the author is the aforementioned relationship to the structure and ratio of commutative planes in $M_2(\bR)$, as well as similar explorations of the structure of maximal commutative subalgebras in the $p$-adic fields $\bQ_p$ for primes $p \neq 2$.


\begin{bibdiv}
    \begin{biblist}
        \bib{Henderson65}{article}{
            title={A short proof of Wedderburn's theorem},
            author={D. W. Henderson},
            journal={Amer. Math. Monthly},
            volume={72},
            date={1965},
            pages={385--386}
        }
        \bib{Serre}{book}{
            title={A Course in Arithmetic},
            author={Jean-Pierre Serre},
            publisher={Springer-Verlag},
            address={New York},
            date={1973},
            pages={5}
        }
        \bib{DandF}{book}{
            title={Abstract Algebra},
            edition={third edition},
            author={D. S. Dummit and R. M. Foote},
            publisher={John Wiley \& Sons, Inc.},
            address={Hoboken, New Jersey},
            date={2004},
            pages={114, 472}
        }
    \end{biblist}
\end{bibdiv}

\end{document}
